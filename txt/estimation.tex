\subsection{Estimation}
\label{estimation}
In Section~\ref{chronology}, modules are considered in relation to each other. 
Thus everything is happy days. However, in this section we consider estimating 
the time a module takes, which proves to be a quite complicated endeavour.

Being able to estimate the time a module takes to complete is useful for the 
author of the module and the module users both. This would be a conservative 
augmentation both in terms of technical and philosophical impact, making it 
perhaps \emph{less interesting} than other prospects explored in this paper, 
but at the same time perhaps all the more immediately useful.

It is easy to conceive of the practical aspects of this idea. There are two 
levels to it. First, let authors of modules estimate the amount of time a 
module will take, and store it as module metadata. The design changes involved 
are quite small, the programming required is minuscule.

There is some benefit to this, but the obvious issue is that the estimator 
might be wrong. It may be wrong in several ways for several reasons. 
Conceivably the estimation reflects what the author is aiming for rather than 
what the author has actually achieved. To put it simply: they might be wrong.

The next logical step then becomes to accumulate how much time users 
\emph{actually} spend. This is more complicated to implement, but not 
\emph{too} complicated. We will require client-side executable code to achieve 
this, which means that any prevention of such code will prevent us from 
gathering useful data. This is not too worrying as there will be little reason 
for users to prevent this code from running, meaning that few users will do 
so. When viewed in isolation, the performance penalties of this code will be 
negligible.

But there are several weaknesses to the metric itself. Naïvely accumulating 
how long a user spends on a particular website results in a plethora of 
useless data. Two easily imagined extrema are users leaving a website up for a 
very long time, and users immediately leaving the website. It is similarly 
easy to imagine why this would happen. As one example of each: 1. Consider a 
primary school pupil visiting a website during the very end of instruction 
time, and leaving it up until the next instruction time (e.g.\ visiting it at 
the end of Friday, and leaving it up through the weekend). 2. Consider a user 
visiting the wrong website and immediately leaving it.

Accounting for those specific problems is non-trivial. Normally, a standard 
deviation cut-off would suffice adequately, but in our case we are faced with 
something like a tri-modal distribution. Maybe it's a leptokurtic 
distribution. That would be nice. But unfortunately, it might be, or it might 
not be. That's a lot worse. And it's about to get worse. Because it might be 
leptokurtic, and then it might change into a fat-headed distribution. And then 
it might reverse back again. And then it might become heavy-tailed. Etc. The 
modules might even have different distributions (that change as more people 
visit them.) If you have a leptokurtic distribution in two modules, a 
heavy-tailed distributed module, and two fat-headed and fat-tailed 
distributions, what then of the composition of these five modules? What then 
when the third becomes leptokurtic and the fourth becomes heavy-tailed? Doing 
this properly is not going to be trivial.

Are we having fun yet? Because it's about to get even more fun. Consider 
having a quiz about monoids in semigroup theory. The author estimates it to 
take $n$ minutes. A primary school student and a maths postgraduate go through 
the quiz. The estimation can in this situation be wrong in two ways --- too 
short and too long. This raises the question --- is the estimation in general 
useful at all?

So then the next step is to tie estimation to knowledge. I.e.\ we need to know 
how much the user knows about something, and derive estimations based on how 
much similarly knowledgeable users know about the same thing. This is 
non-trivial enough to not warrant any more examples. The observant reader 
``gets the picture'', as it were.

Let's take a step back before we end up with a horror story instead of a 
paper. Let's look at the benefits and opportunities afforded to us by 
implementing this.

Authors may receive useful analytics regarding their modules. Some banal 
analytics are ``users are spending longer on this module than you thought they 
would'', and ``these modules in your composition are approximately of the same 
length, but this fourth one takes a lot longer''. The former suffers from the 
context problem, but the latter is actually rather pleasant. Indeed most 
things we can say about one module in relation to some other modules in the 
same composition is usually immediately useful without a Ph.D in statistics.

While module authors are rewarded with feedback, module users are provided 
with useful information on how they want to spend their time, which makes our 
system interrupt flow more seldom. Users may search for modules and 
compositions based on time estimates. The same benefit applies to indirect 
users. E.g.\ classroom teachers might search for the compositions that yields 
the highest ALT to allocated time ratio.

Module authors can be said to be module users as well, in that they will often 
remix modules, and they benefit in a similar manner to module users. They can 
e.g.\ search for modules that fit their estimated composition length.

To make estimates useful for users (including indirect users and remixers) the 
programmer needs a statistics Ph.D or so to deal with transforming the data, 
and then contextualising it. To make estimates useful for authors, they will 
likely need a Ph.D themselves. We should however try our best to help them 
make sense of it. Perhaps authors should have to complete an introduction to 
statistics composition before being able to access their analytics.
