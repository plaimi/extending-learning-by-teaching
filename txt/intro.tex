\section{Introduction}
In an effort to foster learning by teaching, plaimi have previously described 
a canvas for intuitively composing e-learning 
modules\cite{berntsen2015enabling}. This system indirectly emphasises 
chronology, which plaimi previously explored in the tempuhs 
system\cite{berntsen2014tempuhs}.

The canvas system lets users drag and drop e-learning modules onto it, and 
then arrange the data flow of the system, thereby effectively arranging the 
modules into a chronology of modules (a composition of modules). By looking at 
the modules as a chronology, and considering the role time plays in principled 
learning, there are several insights available to us.

This paper describes some such insights. It motivates the insights, and 
explore them in some detail. This includes elaborating and elucidating the 
concepts, as well as giving some notes on their potential implementation. 
There are numerous challenges that the papers elects to not ignore, and seeks 
to mitigate.

Since we are designing a system for learning, it is important that any 
features we consider for inclusion have a sound scientific foundation. The 
insights offered and features discussed are thus considered in a scientific 
context.

There are four angles explored by the paper. Insights afforded by chronicling 
and ordering are presented in Section~\ref{chronology}. Estimation, both by 
way of users estimating their modules' time frame, and by the system 
automatically estimating it, is discussed in Section~\ref{estimation}. Spaced 
repetition as a way of improving retention is explored in 
Section~\ref{repetition}. Finally, features related to synchronous 
collaboration and timeslot synchronisation are described in 
Section~\ref{synchronisation}. We also take the time to make a few remarks 
regarding future research in Section~\ref{further}.
