\section{Conclusion}
Time as a concept is intrinsic to the canvas system proposed by plaimi, as we 
want to maximise ALT. There are several promising insights available to us by 
viewing compositions as chronologies. This simple exercise in perspective 
offers insights that particularly manifest as suggestions we may offer the 
users based on chronology metadata. Knowing which modules tend to follow 
others is simply extremely useful. It is also rather conceptually trivial. It 
does however not directly improve the ALT capabilities of our system per se.

Offering estimation mechanisms in module metadata may alleviate some 
time-management burdening for our users. It makes it easier for authors to 
find modules to fit their composition --- particularly if they are making the 
composition for a situated learning environment in which allocated time must 
be carefully considered --- and it makes it easier for module end-users to 
find suitable learning material. Allowing authors to embed man-made 
estimations as module metadata is a very modest but good extension, but the 
estimations are educated guesses at best. Furthermore, learning material time 
estimations depend heavily on the end-user. Therefore an even better extension 
would be to gather data and do contextual estimation for each user. This is 
however a very difficult problem.

Spaced repetition is accepted as often leading to higher retention and thus 
better learning. We could encode spaced repetition metadata in modules and 
compositions thereof, making our system spaced-repetition-aware, thereby 
further extending its usefulness and area of application. We could also 
implement an insight afforded from spaced repetition software --- the notion 
of self-assessment immediately post-learning. This lets us say useful things 
related to retainment and assessment regardless of spaced repetition.

Collaborative learning can offer a positive learning effect, but if not done 
properly this might be antithetical to ALT. Synchronisation for collaboration 
may be done in realtime, in wait-for-me time, or by timeslots. Realtime 
collaboration in interactive learning material is an interesting prospect. So 
is synchronising users at given intervals, especially when combined with a 
realtime module, e.g.\ evaluation (a quiz or similar) after synchronising the 
users. Timeslots may be a useful way for especially teachers in classroom 
settings to ensure that learners have similar progress. These ideas all 
present difficult problems due to their invasive nature. Further investigation 
is encouraged to take place in a separate system with learning material 
optimised for collaboration, which might to some extent marry the advantages 
of ALT and collaborative learning.

To sum up, the following features should be implemented:
\begin{itemize*}
  \item order-awareness for chronology insights,
  \item author estimation of module length,
  \item a metadata framework for spaced repetition,
  \item and post-module self-assessment capabilities.
\end{itemize*}

These are, not coincidentally, the most modest features proposed in the paper.

The following more invasive changes were discussed:

\begin{itemize*}
  \item system estimation of module length,
  \item contextualised (user-customised) estimation of module length,
  \item spaced-repetition encouraging,
  \item collaborative realtime module use,
  \item easy referring to modules and specific elements therein,
  \item wait-for-me collaboration in which users are periodically synchronised 
  to the same module,
  \item timeslot collaboration to ensure that users are somewhat synchronised,
  \item various combinations and nesting of the proposed synchronisation 
  methods,
  \item instructor-integration mechanics for collaborative learning,
  \item and a sub-system optimised for collaborative learning.
\end{itemize*}

Further research is encouraged for all of these more invasive features. 
Collaborative learning is particularly interesting due to the weight of its 
potential augmentation. It is also particularly difficult due to its potential 
negative impact on ALT.
